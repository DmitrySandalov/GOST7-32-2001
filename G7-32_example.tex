\documentclass[utf8,usehyperref]{G7-32}

 \TableInChaper % таблицы будут нумероваться в пределах раздела
 %\PicInChaper   % рисунки будут нумероваться в пределах раздела
 \setlength\GostItemGap{2mm}% для красоты можно менять от 0мм

 % Определяем заголовки для титульной страницы
 \NirOrgLongName{Самара}{НИИ Всякой Фигни}
 \NirBoss{Директор}{П.~П.~Петров}
 \NirManager{Начальник отдела}{И.~И.~Иванов}

\begin{document}
\frontmatter
 \NirTitlePage{111}{Исследование всякой фигни}
 \Executors Здесь список исполнителей
 \Referat
 Отчет \totalpages~с, \totaltables~таблица, \totalfigures~рисунок, \totalbibs~источник.
 \tableofcontents
 \NormRefs Здесь некоторое количество нормативных ссылок
 \Defines Определения всякие
 \Abbreviations
 \begin{abbreviation}
  \item[ПБЗ] пункты без заголовка
  \item[ВНЧ] вообще непонятно что
 \end{abbreviation}

 \Introduction Вступительное слово
\mainmatter

\chapter{О разном}
\ttl
\section{О всяком разном}
\ttl
\subsection{Это уж просто безобразие}
\ttl
\subsubsection{А это~--- форменное безобразие}\label{L1}
Ну и что?
\begin{enumerate}
 \item\label{L1:I1}
  а ничего;
 \item
  а именно:
  \begin{enumerate}
   \item
    совсем ничего. Я бы даже сказал~--- вс? это полный бред
    и попытка добиться переноса пункта перечисления на новую строчку;
   \item
    абсолютно ничего~\cite{bib:test1}.
  \end{enumerate}
\end{enumerate}

\section{О другом}
\nsubsection Случаются и ПБЗ.

И что, спросите вы? Уже писали~--- \pref{L1}{L1:I1}.

Однако, воткн?м тута табличку.

\begin{longtable}{|c|c|p{110mm}|}
 \multicolumn{3}{l}{\tablename~\thetable~---~Всякая фигня\label{T:T1}}\\\hline
 Что & Куда & Откуда \\\hline
\endfirsthead
 \multicolumn{3}{l}{Продолжение таблицы~\ref{T:T1}}\\\hline
 Что & Куда & Откуда \\\hline
\endhead
 Это & фиг знает & ну, если рассуждать здраво,  то х.е.з. \\\hline
 То  & то же     & а это ВНЧ \\\hline
\end{longtable}

Ну и без рисунка \ref{fig1} нам никак не обойтись:

\begin{figure}
 \caption{Так себе рисуночек}\label{fig1}
 \centering{
  \begin{picture}(100,50)
   \put(  0, 0){\line( 1, 0){100}}
   \put(  0, 0){\line( 0, 1){ 50}}
   \put(100,50){\line(-1, 0){100}}
   \put(100,50){\line( 0,-1){ 50}}
  \end{picture}
 }
\end{figure}

\backmatter
 \Conclusion на последок
 \begin{thebibliography}{99}
  \bibitem{bib:test1}
  Нечто изданное некогда.
 \end{thebibliography}
\end{document}
