% This package designed and commented in russian utf8 encoding.
%
% Класс документов по ГОСТ 7.32-2001 "Отч?т о научно-исследовательской работе"
% на основе ГОСТ 2.105-95
% Автор - Алексей Томин, с помощью списка рассылки latex-gost-request@ice.ru,
%  "extreport.cls", "lastpage.sty" и конференции RU.TEX
% Лицензия GPL
% Все вопросы, замечания и пожелания сюда: mailto:alxt@yandex.ru
% Дальнейшая разработка и поддержка - Михаил Конник,
% связаться можно по адресу mydebianblog@gmail.com

\documentclass[utf8,usehyperref,12pt]{G7-32}
\usepackage[T2A]{fontenc}
\usepackage[utf8]{inputenc} %% ваша любимая кодировка здесь
\usepackage[english,russian]{babel} %% это необходимо для включения переносов
\usepackage{float}
\usepackage[dvips]{graphicx}
\graphicspath{{pictures/}}

\TableInChaper % таблицы будут нумероваться в пределах раздела
\PicInChaper   % рисунки будут нумероваться в пределах раздела
\setlength\GostItemGap{2mm}% для красоты можно менять от 0мм

% Определяем заголовки для титульной страницы
\NirOrgLongName{\textsc{ООО <<Рога и Копыта>>}} %% Полное название организации

\NirBoss{Директор ООО <<Рога и Копыта>>}{И.И.Иванов} %% Заказчик, утверждающий НИР
\NirManager{доцент, к.т.н.}{К.К.Петров} %% Название организации

\NirYear{2020}%% если нужно поменять год отч?та; если закомментировано, ставится текущий год
\NirTown{г. Москва,} %% город, в котором написан отч?т
% по проекту \No8550: 

% \NirIsAnnotacion{АННОТАЦИОННЫЙ } %% Раскомментируйте, если это аннотационный отч?т

\NirUdk{УДК \No 2123132123}
\NirGosNo{Регистрационный \No 123123}

\NirStage{Этап \No 1.1}{промежуточный}{<<Обзор современного состояния торсионных наногенераторов>>} %%% Этап НИР: {номер этапа}{вид отч?та - промежуточный или заключительный}{название этапа}

\bibliographystyle{unsrt} %Стиль библиографических ссылок БибТеХа

%%%%%%%<------------- НАЧАЛО ДОКУМЕНТА
\begin{document}
\usefont{T2A}{ftm}{m}{} %%% Использование шрифтов Т2 для возможности скопировать текст из PDF-файлов.

\frontmatter %%% <-- это выключает нумерацию ВСЕГО; здесь начинаются ненумерованные главы типа Исполнители, Обозначения и прочее

\NirTitle{\textbf{<<Торсионные наногенераторы плазменных стволовых клеток с протонной накачкой>>}} %%% Название НИР и генерация титульного листа


\Executors %% Список исполнителей здесь
%% это рисует линию размера 3мм и толщиной 0.1 пункт
\begin{longtable}{p{0.35\linewidth}p{0.2\linewidth}p{0.35\linewidth}}
Научный руководитель, 	&		&	\\
доцент К.К.Петров	&\rule{1\linewidth}{0.1pt}	&  \\ \vspace{1cm}

с.н.с, к.т.н,  &		&	\\
Ж.Ж. Балбесов, & \rule{1\linewidth}{0.1pt}& \\
\end{longtable}

\Referat %% Реферат отч?та, не более 1 страницы
В соответствии с календарным планом проекта \No, настоящий аннотационный отч?т содержит итоги работ по подэтапу 1.1 выполнения НИОКР ``Обзор современного состояния торсионных наногенераторов''.

На данном этапе проводись работы по подбору и приобретению спецоборудования, необходимого для выполнения НИОКР, теоретические и экспериментальные иссл.........

Расчетно и экспериментально обоснован выбор оптической схемы с торсионным излучением трансформируемой частичной пространственной когерентности в качестве...

В результате работ по подбору оборудования определена элементная база для аппаратной реализации основных узлов торсионного наногенератора. Подбор оборудования обуславливал........

Выполнено математическое моделирование ряда методов .... 

Показано, что возможно использование вейвлет-преобразования........

Сформулированы требования к управляющему....

Полученные результаты создают основу для выполнения работ по второму этапу договора, предполагающих .........

\tableofcontents

\NormRefs % Нормативные ссылки 
\Defines % Необходимые определения 


\Abbreviations %% Список обозначений и сокращений в тексте
\begin{abbreviation}
\item[ТНГ] Торсионный нано генератор -- образец лженаучного волюнтаризма.
\end{abbreviation}

\Introduction
Большинство современных систем торсионного\cite{filosofyNewestdict} генерирования...

\mainmatter %% это включает нумерацию глав и секций в документе ниже

\chapter{Наногенераторы торсионных полей как вечный двигатель прогресса}
Используйте окружения chapter и section как обычно. Вообще, набор текста в этом шаблоне ничем не отличается от других.

\section{Проблематика лженаучного мышления}
\subsection{Пример торсионных недонаногенераторов}

Для вставки рисунков используйте стандартное окружение figure и директиву includegraphics (в этом примере это закомментировано). Пустой рисунок привед?н на рис.~\ref{fig1}.

\begin{figure}
 \caption{Пример рисунка}\label{fig1}
 \centering{
  \begin{picture}(100,50)
   \put(  0, 0){\line( 1, 0){100}}
   \put(  0, 0){\line( 0, 1){ 50}}
   \put(100,50){\line(-1, 0){100}}
   \put(100,50){\line( 0,-1){ 50}}
% \includegraphics[width=0.5\linewidth]{picture.eps} }
  \end{picture}
 }
\end{figure}

В отч?тах могут быть и таблицы - см.табл.~\ref{T:T1}.

\begin{longtable}{|c|c|p{110mm}|}
 \multicolumn{3}{l}{\tablename~\thetable~---~Пример таблицы\label{T:T1}}\\\hline
 Название 1  & Название 2 & Название 3 \\
\hline
\endfirsthead
 \multicolumn{3}{l}{Продолжение таблицы~\ref{T:T1}}\\
\hline
1 & 2 & 3 \\
\hline
\endhead
Это  & пример & данных  \\
\hline
помещ?нных & внутрь & таблицы \\
\hline
\end{longtable}


\backmatter %% Здесь заканчивается нумерованная часть документа и начинаютяс заключение и ссылки

\Conclusion % заключение к отч?ту

\begin{thebibliography}{1} %% здесь библиографический список

\bibitem{filosofyNewestdict}
{Грицанов} А.А.~и др.
\newblock {\em Новейший философский словарь}.
\newblock Мн.: Книжный Дом., 2003.

\end{thebibliography}

% \bibliography{biblio/filosofy} %% вместо вставки библиографии можно использовать базы BiBTeX - просто раскомментируйте эту строку.
\end{document}
